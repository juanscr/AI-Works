\documentclass[conference]{IEEEtran}
\usepackage[english]{babel}
\usepackage{geometry}
\usepackage{amsmath}
\usepackage{amsthm}
\usepackage{graphicx}
\usepackage{caption}
\usepackage[utf8]{inputenc}


%%%%%%%% SUB-FIGURE PACKAGE
\usepackage{subcaption}

%%%%%%%% MULTI-COLUMNS PACKAGE
\usepackage{multicol}

%%%%%%%% PERSONAL COMMANDS
\usepackage{amssymb}

%%%% Important sets
\renewcommand{\O}{\mathbb{O}}
\newcommand{\N}{\mathbb{N}}
\newcommand{\Z}{{\mathbb{Z}}}
\newcommand{\Q}{{\mathbb{Q}}}
\newcommand{\R}{{\mathbb{R}}}

%%%% Usual operations
\newcommand{\pow}[2]{#1^{#2}}
\newcommand{\expp}[1]{e^{#1}}
\newcommand{\fst}{\mathrm{fst}}
\newcommand{\snd}{\mathrm{snd}}

%%%% Lambda Calculus
\newcommand{\dneq}{\,\, \# \,\,}
\newcommand{\prm}[1]{\mathrm{\mathbf{#1}}}
\renewcommand{\S}{\prm{S}}
\newcommand{\I}{\prm{I}}
\newcommand{\K}{\prm{K}}
\newcommand{\ch}[1]{\ulcorner #1 \urcorner}

%%%% Ordinal Lambda Calculus
\newcommand{\ordAlph}{\Sigma_{\text{Ord}}}
\newcommand{\termOrd}{\text{Term}_\text{Ord}}
\newcommand{\fl}{\mathrm{fl}}
\newcommand{\sk}{\mathrm{sk}}

%% Superscript to the left
% https://latex.org/forum/viewtopic.php?t=455
\usepackage{tensor}
\newcommand{\app}[3]{\tensor*[^{#1}]{\left(#2, #3\right)}{}}

%%%% Make optional parameter
% https://bit.ly/3jVGRwQ
\usepackage{xparse}

%%%% Statistics
\NewDocumentCommand{\E}{o m}{
  \IfNoValueTF{#1}
  {\mathbb{E}\left[#2\right]}
  {\mathbb{E}^{#1}\left[ #2\right]}
}
\NewDocumentCommand{\V}{o m}{
  \IfNoValueTF{#1}
  {\mathrm{Var}\left[#2\right]}
  {\mathrm{Var}^{#1}\left[ #2\right]}
}
\RenewDocumentCommand{\P}{o o m}{
  \IfNoValueTF{#1}
  {\IfNoValueTF{#2}
    {\mathrm{P}\left(#3\right)}
    {\mathrm{P}^{#2}\left(#3\right)}}
  {\IfNoValueTF{#2}
    {\mathrm{P}_{#1}\left(#3\right)}
    {\mathrm{P}_{#1}^{#2} \left(#3\right)}}
}

%%%% Lambda Calculus
\NewDocumentCommand{\cx}{o}{
  \IfNoValueTF{#1}
  {\left[\quad\right]}
  {\left[\, #1 \,\right]}
}

%%%% Create absolute value function
% https://bit.ly/33Rkq6H
\usepackage{mathtools}
\DeclarePairedDelimiter\abs{\lvert}{\rvert}%
\DeclarePairedDelimiter\norm{\lVert}{\rVert}%
\makeatletter
\let\oldabs\abs
\def\abs{\@ifstar{\oldabs}{\oldabs*}}
%
\let\oldnorm\norm
\def\norm{\@ifstar{\oldnorm}{\oldnorm*}}
\makeatother

%%%%%%%% LOGIC TREES
\usepackage{prftree}

%%%%%%%% SPLIT EQUATIONS
% https://bit.ly/33P1OUM
\allowdisplaybreaks

%%%%%%%% FLOAT SPECIFIER
% https://bit.ly/30Wi4BC
\usepackage{float}

%%%%%%%% TO USE SHORT COMMANDS FOR VECTOR LINES
\usepackage{esvect}

%%%%%%%% DIFFERENT FONTS FOR MATH
\usepackage{mathrsfs}

%%%%%%%% FOOTNOTE STUFF
\renewcommand{\thefootnote}{\fnsymbol{footnote}}



%%%%%%%% MARGIN
\geometry{verbose, letterpaper, tmargin=3cm,
  bmargin=3cm,lmargin=2.5cm,rmargin=2.5cm}

%%%%%%%% PARAGRAPH SETTINGS
% https://bit.ly/36WrtN4
\setlength\parindent{0pt}

% https://bit.ly/371dvto
\setlength{\parskip}{5pt}

%%%%%%%% HYPERREF PACKAGE
\usepackage{hyperref}
\hypersetup{linkcolor=blue}
\hypersetup{citecolor=blue}
\hypersetup{urlcolor=blue}
\hypersetup{colorlinks=true}


%%%%%%%% BIB-LATEX STUFF
\usepackage[style=ieee,
            bibstyle=ieee,
            citestyle=ieee,
            hyperref=true,
            backend=biber]{biblatex}
\addbibresource{ref.bib} %Put relative path to ref

%%%%%%%% DEFINITION AND THEOREM DEFINITIONS
\theoremstyle{definition}
\newtheorem{definition}{Definition}[section]

\theoremstyle{remark}
\newtheorem{remark}{Remark}

\theoremstyle{remark}
\newtheorem{question}{Question}

\newtheorem{theorem}{Theorem}[section]


%%%%%%%% CODE RENDERING !!! UNCOMMENT IF NEEDED !!!
% Compile with flag -shell-escape
%\usepackage{minted}

%%%%%%%% START DOCUMENT
\title{Supervised Learning for Psychological Attention to High School Students}

\author{\IEEEauthorblockN{Juan S. Cárdenas-Rodríguez}
  \IEEEauthorblockA{\textit{Department of Mathematical Sciences} \\
    \textit{EAFIT University}\\
    Medellín, Colombia \\
    jscardenar@eafit.edu.co} \and \IEEEauthorblockN{David Plazas}
  \IEEEauthorblockA{\textit{Department of Mathematical Sciences} \\
    \textit{EAFIT University}\\
    Medellín, Colombia \\
    dplazas@eafit.edu.co} }


\begin{document}
\maketitle

%%%%%%%%%%%%%%%%%%%%%%%%%%%%%
\begin{abstract}
  Hello.
\end{abstract}

\begin{IEEEkeywords}
  Nice.
\end{IEEEkeywords}
%%%%%%%%%%%%%%%%%%%%%%%%%%%%%

\textit{Note}: All the code and the latex file can be found in
\href{https://github.com/juanscr/ai-works}{the GitHub repository}.

\section{Introduction}

Mental health is crucially important for every human being and has become one of
the most important issues surrounding human health. According to the World
Health Organization (WHO)\footnote{WHO's \href{https://bit.ly/34l2v94}{2001
    report.}} report of 2001, mental disorders can affect 25\% of the
population, therefore, suggesting the ongoing trend of people suffering from
lack of mental health. Additionally, with the recent events caused by the
COVID-19 pandemic, mental issues have risen as consequence of lockdowns all
around the globe, as have been proved by recent studies
\parencite{rossi2020,xiong2020}.

Furthermore, mental health is more important to monitor in adolescents and kids
as this group is in the process of developing their mental structure and
character. In this manner, as they continue creating their view about the world
they must be surrounded by a healthy environment that promotes healthy habits
and relationships with their surroundings.

As a result, schools and parents have an important responsibility to the
students of creating this environment and wisely handling mental health issues.
However, high schools do not always know how to handle or detect cases of mental
health issues. Additionally, technology has led to targeted harassing becoming
more popular over time, therefore, difficulting the probability of detecting
these cases.

Some of the most common and important mental health issues are anxiety,
depression, and hyperkinetic disorder. According to \textcite{schulte2016},
around 10\% to 20\% of children suffer from hyperkinetic disorder. Additionally,
school teachers just assume this disorder is a cause of undisciplined students
and can take the hyperkinetic student to a depressive state. These three mental
disorders can leave severe damage to a kid and can lead them to suicide.

Hence, it is of the utmost importance to quickly and efficiently detect students
that might be having a mental health issue to immediately start a process with
him/her. Furthermore, this detection has to be done with measurable variables
that can be easily obtained from questionnaires or school data.

The present work is oriented to the application of supervised learning
techniques (decision trees, support vector machines and neural networks) for
classifying the data from the survey, aiming for the construction of a model to
detect individuals with mental health issues. Recent developments and the
increasing popularity of supervised techniques has contributed to their
application in different areas, among them, mental health.

The novel work by \textcite{thieme2020} makes and excellent literature review of
Machine Learning (ML) in mental health, including the current state-of-the-art
and formulates new strategies for integrating human-oriented research with
multi-disciplinary. They also conclude that ML applied in mental health is
``still in its infancy'' given the complexity and difficulties of constructing
robust models for clinically reliable outputs.

Additionally, the work presented by \textcite{shatte2019} is also a complete
compendium of recent ML applications in mental health, including the usage
of Support Vector Machines (SVMs), Decision Trees (DTs), Neural Networks (NN)
and beyond. This paper includes a systematic review of over three hundred papers
and it concludes that these techniques have brought benefits 
across the areas of diagnosis, treatment, research and clinical administration.

Finally, another overview of applications of Artificial Intelligence (AI) for Mental 
Illnesses is the research published by \textcite{graham2019}, where the authors
reviewed 28 research papers which include modern methodologies (and 
technologies) for predicting or classifying psychological illnesses such as 
depression, schizophrenia or suicidal behavior. The study discusses how the
application of AI techniques supports the clinical practice while considering its
limitations, it also presents some ethical implications and states which areas need
further development. 

This document is organized as
follows: Section \ref{sec:meth} shows the general methodology used in this work,
moreover Section \ref{sec:res} shows the obtained results over the data and
slight modifications, finally Section \ref{sec:conc} concludes this work and
sets pillars for future works.

\section{Methodology}\label{sec:meth}
All the code developed for this work was implemented in Julia 1.5.3. and 
can be found in the authors'
\href{https://github.com/juanscr/ai-works}{GitHub repository}.

\subsection{Survey}
The used survey for extracting data was initially designed for developing
a Fuzzy inference expert system, but the data was labeled according to
unsupervised clustering techniques, obtaining 2 clusters through the subtractive
algorithm \parencite{chiu1994} and then sharpened using Fuzzy-C-Means 
\parencite{dunn1973}. The data has 6 variables and 113 observations.

\subsection{Support Vector Machines}
We refer to \textcite{burges1998} for the theoretical background on Support
Vector Machines (SVMs).

\subsection{Decision Trees}
We refer to 

\subsection{Neural Networks}
We refer to \textcite{aggarwal2018} for the formulation of the Neural Netwok 
(NN) learning machine and the details of its implementation, such as the 
backpropagation algorithm for training and the bias correction. The NN was
implemented by the authors, including the bias correction and the 

\section{Results}\label{sec:res}

\section{Conclusions}\label{sec:conc}

\printbibliography
\end{document}
