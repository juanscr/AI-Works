\documentclass[fleqn]{article}
\usepackage[spanish]{babel}
\usepackage{amsmath}
\usepackage{amsthm}
\usepackage{graphicx}
\usepackage[utf8]{inputenc}

%%%%%%%% MARGIN
\usepackage[left=1in, right=1in, top=0.8in, bottom=0.8in]{geometry}

%%%%%%%% NO PARAGRAPH INDENT
% https://tex.stackexchange.com/questions/27802/set-noindent-for-entire-file
\setlength\parindent{0pt}

%%%%%%%% SUB-FIGURE PACKAGE
\usepackage{subcaption}

%%%%%%%% HYPERREF PACKAGE
\usepackage{hyperref}
\hypersetup{linkcolor=blue}
\hypersetup{citecolor=blue}
\hypersetup{urlcolor=blue}
\hypersetup{colorlinks=true}

%%%%%%%% MULTI-COLUMNS PACKAGE
\usepackage{multicol}

%%%%%%%% SETS DEFINITIONS
\usepackage{amssymb}
%%%% Important sets
\renewcommand{\O}{\mathbb{O}}
\newcommand{\N}{\mathbb{N}}
\newcommand{\Z}{{\mathbb{Z}}}
\newcommand{\Q}{{\mathbb{Q}}}
\newcommand{\R}{{\mathbb{R}}}

%%%% Statistics
\newcommand{\E}[1]{\mathbb{E}\left[#1 \right]}
\newcommand{\V}[1]{\mathrm{Var}\left[#1 \right]}

%%%% Lambda Calculus Symbols
\newcommand{\dneq}{\,\, \# \,\,}
\renewcommand{\S}{\pmb{\mathrm{S}}}
\newcommand{\I}{\pmb{\mathrm{I}}}
\newcommand{\K}{\pmb{\mathrm{K}}}
\newcommand{\ch}[1]{\ulcorner #1 \urcorner}

%%%% Ordinal Lambda Calculus Symbols
\newcommand{\ordAlph}{\Sigma_{\text{Ord}}}
\newcommand{\termOrd}{\text{Term}_\text{Ord}}
\newcommand{\fl}{\mathrm{fl}}
\newcommand{\sk}{\mathrm{sk}}

%%%% Superscript to the left
% https://latex.org/forum/viewtopic.php?t=455
\usepackage{tensor}
\newcommand{\app}[3]{\tensor*[^{#1}]{\left(#2, #3\right)}{}}

%%%% Make optional parameter
% https://tex.stackexchange.com/questions/217757/special-behavior-if-optional-argument-is-not-passed
\usepackage{xparse}
\NewDocumentCommand{\cx}{o}{
  \IfNoValueTF{#1}
  {\left[\quad\right]}
  {\left[\, #1 \,\right]}
}

%%%%%%%% LOGIC TREES
\usepackage{prftree}

%%%%%%%% SPLIT EQUATIONS
% https://tex.stackexchange.com/questions/51682/is-it-possible-to-pagebreak-aligned-equations
\allowdisplaybreaks

%%%%%%%% CODE RENDERING
% Compile with flag -shell-escape
% \usepackage{minted}

%%%%%%%% EXAM PACKAGE
\usepackage{mathexam}

%%%%%%%% CHANGE MARGINS ITEMIZE
\usepackage{enumitem}

\usepackage{float}

%%%%%%%% START DOCUMENT

\ExamClass{INTELIGENCIA ARTIFICIAL}
\ExamName{EXAMEN 1}
\ExamHead{\today}

\let\ds\displaystyle

\begin{document}
 \vspace{0.3cm}
   % Information of the student
   \begin{itemize}[leftmargin=6.25cm, labelsep=0.5cm]

     \item[\textit{Nombre}] \scalebox{1.2}{Juan Sebastián Cárdenas Rodríguez} % Name
     \item[\textit{Código}] 201710008101 % Code

   \end{itemize}

   \begin{enumerate}
     \item Pregunta 1.
       \begin{itemize}
         \item Medio excluido:
           \begin{figure}[H]
             \centering
             \includegraphics[scale=.5]{figs/1a}
           \end{figure}
        \pagebreak
         \item D'Morgan 1:
           \begin{figure}[h]
             \centering
             \includegraphics[scale=.5]{figs/1b1}
           \end{figure}
           \begin{figure}[h]
             \centering
             \includegraphics[scale=.5]{figs/1b2}
           \end{figure}
           \pagebreak
         \item D'Morgan 2:
           \begin{figure}[H]
             \centering
             \includegraphics[scale=.5]{figs/1c}
           \end{figure}
       \end{itemize}

     \item Pregunta 2.
     \begin{itemize}
       \item Producto Algebraico.
       \begin{figure}[H]
         \centering
         \includegraphics[scale=.5]{figs/2a1}
       \end{figure}
       \begin{figure}[H]
         \centering
         \includegraphics[scale=.5]{figs/2a2}
       \end{figure}

       \item Producto acotado.
       \begin{figure}[H]
         \centering
         \includegraphics[scale=.5]{figs/2b1}
       \end{figure}
       \begin{figure}[H]
         \centering
         \includegraphics[scale=.8]{figs/2b2}
       \end{figure}
       \begin{figure}[H]
         \centering
         \includegraphics[scale=.8]{figs/2b3}
       \end{figure}
       \begin{figure}[H]
         \centering
         \includegraphics[scale=.5]{figs/2b4}
       \end{figure}
       \item Desigualdades:
       \begin{figure}[H]
         \centering
         \includegraphics[scale=.5]{figs/2c1}
       \end{figure}
       \begin{figure}[H]
         \centering
         \includegraphics[scale=.5]{figs/2c2}
       \end{figure}
       \begin{figure}[H]
         \centering
         \includegraphics[scale=.5]{figs/2c3}
       \end{figure}
       \begin{figure}[H]
         \centering
         \includegraphics[scale=.5]{figs/2c4}
       \end{figure}
     \end{itemize}

     \item Pregunta 3:
     \begin{itemize}
       \item Entradas:
       \begin{enumerate}
         \item Socialización emocional de un estudiante.
         \item Rendimiento académico.
         \item Nivel de atención en clase.
       \end{enumerate}
       \item Salidas:
       \begin{enumerate}
         \item Nivel de ansiedad.
         \item Nivel de hiperactividad o déficit de atención.
         \item Nivel de depresión.
       \end{enumerate}
       \item Universo discurso:
       \begin{enumerate}
         \item Socialización: Un rango entre 0 y 40.
         \item Rendimiento académico: Un rango entre 0 y 5.
         \item Nivel de atención en clase: Un rango entre 0 y 3.
         \item Los niveles de la enfermedad tienen un universo discurso entre 0
           y 100.
       \end{enumerate}
     \end{itemize}
   \end{enumerate}
\end{document}
